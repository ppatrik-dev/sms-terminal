% dokumentacia

\documentclass[11pt]{article}
\usepackage[a4paper, text={17cm, 24cm}, left=2cm, top=3cm]{geometry}
\usepackage[utf8]{inputenc}
\usepackage[T1]{fontenc}
\usepackage[slovak]{babel}
\usepackage{csquotes}
\usepackage[backend=biber,style=numeric]{biblatex}
\addbibresource{references.bib}
\usepackage[slovak,ruled,vlined]{algorithm2e}
\addbibresource{zdroje.bib}
\usepackage{graphicx}
\usepackage{float}
\usepackage{blindtext}
\setlength{\parindent}{0pt}
\usepackage[hidelinks]{hyperref}
\usepackage{url} 
\usepackage{tikz}
\usetikzlibrary{positioning}
\usepackage{caption}

\begin{document}

\begin{titlepage}
    \begin{center}
        \Huge\textsc{Vysoké učenie technické v Brne}\\
        \huge\textsc{Fakulta informačných technógií}\\
        \vspace{\stretch{0.382}}
        \LARGE Dokumentácia k projektu z predmetu IMP\\
        \huge \textbf{Písanie na displeji maticovou klávesnicou}\\
        \vspace{\stretch{0.618}}
    \end{center}
    
    \begin{minipage}{0.4 \textwidth}
        \Large \today 
    \end{minipage}
        \hfill
    \begin{minipage}[r]{0.4 \textwidth}
        \Large Patrik Procházka (xprochp00)
    \end{minipage}
\end{titlepage}

\newpage

\tableofcontents
\newpage

\section{Úvod}
Cieľom projektu bol návrh a implementácia textového terminálu na platforme \textsc{ESP32}, ktorý simuluje písanie textových správ na starých tlačítkových telefónoch. Zariadene slúži na písanie krátkych textových správ (\textsc{SMS}) s využitím maticovej klávesnice a grafického displeja.\newline

Projekt bol implementovaný v prostredí \texttt{Arduino IDE}, avšak s dôrazom na kvalitu riešenia. Aplikácia okrem základného snímania kláves a cyklického zadávania znakov, obsahuje ďalšiu funkcionalitu ako napríklad:
\begin{itemize}
    \item \textbf{Vstupné režimy~--} Prepínanie medzi malými písmenami, veľkými písmenami a inteligentným režimom, ktorý automaticky kapitalizuje začiatky viet.
    \item \textbf{Správa displeja~--} Implementácia stránkovania pre text správy dlhší ako kapacita obrazovky, vizualizácia kurzora s rešpektovaním okrajov displeja a zobrazenie panelu s aktuálnymi informáciami.
    \item \textbf{Navigácia a editácia~--} Logika posunu kurzora v štyroch smeroch v rámci správy s možnosťou spätnej uprávy správy.
    \item \textbf{Interaktívna nápoveda~--} Zobrazenie menu ovládania terminálu podržaním funkčnej klávesy~\texttt{*}.
\end{itemize}

Kľúčovým technickým aspektom bola požiadavka na \textbf{manuálne} vyčítanie maticovej klávesnice bez použitia knižnice. Tento proces využívajúci \textsc{GPIO} vstupy a výstupy mikrokontroléra je popísaný v algoritme \ref{alg:keypad_alg2e}.
Výsledkom je plne funkčný prototyp terminálu, ktorý umožňuje užívateľovi interaktívne písať textové \textsc{SMS} správy. 

\newpage
\section{Technická realizácia}
Na realizáciu terminálu bol použitý mikrokontrolér \textsc{ESP32}, konkrétne model WeMos \textsc{D1 R32} \cite{AZDeliveryD1R32}, ktorý zabezpečuje spracovanie vstupov a komunikáciu s displejom. Jednotlivé komponenty boli zapojené na nepájivom poli (angl. \textit{breadboard}), čo umožnilo rýchlu prototypovú realizáciu zapojenia. Používateľský vstup je realizovaný pomocou maticovej klávesnice, zatiaľ čo výstupné informácie sú zobrazované na displeji.

\subsection{Klávesnica}
Na používateľský vstup bola použitá maticová klávesnica s troma stĺpcami a štyrmi riadkami \cite{farnell1662617}. Na obrázku \ref{fig:keypad_matrix} je zobrazené rozloženie jednotlivých kláves a v tabuľke \ref{tab:output_arrangement} je zobrazené mapovanie výstupných pinov klávesnice na jednotlivé riadky a stĺpce.

\begin{center}
    \nopagebreak
    % This picture was generated by AI.
    \begin{minipage}[b]{0.52\textwidth}
        \centering
        \begin{tikzpicture}[
            scale=0.7, transform shape, 
            key/.style={circle, draw, thick, minimum size=7mm, fill=white}, 
            line/.style={thick},
            label/.style={font=\footnotesize\sffamily\bfseries}
        ]
        % --- Keys ---
        \node[key] (k1)    at (0,3.6) {1}; \node[key] (k2)    at (1.5,3.6) {2}; \node[key] (k3)    at (3,3.6) {3};
        \node[key] (k4)    at (0,2.4) {4}; \node[key] (k5)    at (1.5,2.4) {5}; \node[key] (k6)    at (3,2.4) {6};
        \node[key] (k7)    at (0,1.2) {7}; \node[key] (k8)    at (1.5,1.2) {8}; \node[key] (k9)    at (3,1.2) {9};
        \node[key] (kstar) at (0,0.0) {*}; \node[key] (k0)    at (1.5,0.0) {0}; \node[key] (khash) at (3,0.0) {\#};

        % --- Connections ---
        \draw[line] (0, 4.2) -- (k1) -- (k4) -- (k7) -- (kstar) -- (0, -0.6);
        \draw[line] (1.5, 4.2) -- (k2) -- (k5) -- (k8) -- (k0) -- (1.5, -0.6);
        \draw[line] (3, 4.2) -- (k3) -- (k6) -- (k9) -- (khash) -- (3, -0.6);
        \draw[line] (-0.6, 3.6) -- (k1) -- (k2) -- (k3) -- (4.0, 3.6);
        \draw[line] (-0.6, 2.4) -- (k4) -- (k5) -- (k6) -- (4.0, 2.4);
        \draw[line] (-0.6, 1.2) -- (k7) -- (k8) -- (k9) -- (4.0, 1.2);
        \draw[line] (-0.6, 0.0) -- (kstar) -- (k0) -- (khash) -- (4.0, 0.0);

        % --- Labels ---
        \node[label, right] at (4.1, 3.6) {R1}; \node[label, right] at (4.1, 2.4) {R2};
        \node[label, right] at (4.1, 1.2) {R3}; \node[label, right] at (4.1, 0.0) {R4};
        \node[label, below] at (0, -0.7) {C1}; \node[label, below] at (1.5, -0.7) {C2};
        \node[label, below] at (3, -0.7) {C3};
        \end{tikzpicture}
        \captionof{figure}{Schéma rozloženia klávesnice}
        \label{fig:keypad_matrix}
    \end{minipage}
    \hfill
    \begin{minipage}[b]{0.42\textwidth}
        \centering
        \small
        \begin{tabular}{|c|c|}
        \hline
        \textbf{Pin} & \textbf{Riadok/Stĺpec} \\
        \hline
        1 & COL 2 \\
        2 & ROW 1 \\
        3 & COL 1 \\
        4 & ROW 4 \\
        5 & COL 3 \\
        6 & ROW 3 \\
        7 & ROW 2 \\
        \hline
        \end{tabular}
        \captionof{table}{Mapovanie výstupných pinov}
        \label{tab:output_arrangement}
    \end{minipage}
\end{center}

Jednotlivé piny klávesnice pre riadky a stĺpce boli následne pripojené na \textsc{GPIO} (General Purpose Input/Output) piny mikrokontroléra, konkrétne čísla pinov je možné vidieť v tabuľkách \ref{tab:keypad_cols} a \ref{tab:keypad_rows}.

\begin{table}[h!]
\centering
\begin{minipage}[b]{0.45\textwidth}
\centering
\begin{tabular}{|c|c|}
\hline
\textbf{Stĺpec} & \textbf{GPIO pin} \\
\hline
COL 1 & 25 \\
COL 2 & 26 \\
COL 3 & 13 \\
\hline
\end{tabular}
\caption{GPIO piny pre stĺpce klávesnice}
\label{tab:keypad_cols}
\end{minipage}
\hfill
\begin{minipage}[b]{0.45\textwidth}
\centering
\begin{tabular}{|c|c|}
\hline
\textbf{Riadok} & \textbf{GPIO pin} \\
\hline
ROW 1 & 14 \\
ROW 2 & 12 \\
ROW 3 & 19 \\
ROW 4 & 16 \\
\hline
\end{tabular}
\caption{GPIO piny pre riadky klávesnice}
\label{tab:keypad_rows}
\end{minipage}
\end{table}

\subsection{Displej}
Na zobrazovanie správy bol použitý \textsc{OLED} displej s uhlopriečkou 0,96 palca \cite{Dragonwake_MSP096X}. \textbf{Synchrónna sériová} komunikácia medzi mikrokontrolérom a displejom prebieha pomocou protokolu \textsc{SPI}. V nasledujúcej tabuľke \ref{tab:spi_pins} je zobrazené mapovanie \textsc{SPI} pinov displeja na \textsc{GPIO} piny mikrokontroléra.

\begin{table}[h!]
\centering
\begin{tabular}{|l|c|}
\hline
\textbf{SPI signál} & \textbf{GPIO pin} \\
\hline
CS        & 5  \\
DC        & 27 \\
RES       & 17 \\
SCLK (D0) & 18 \\
MOSI (D1) & 23 \\
\hline
\end{tabular}
\caption{Kofigurácia pinov displeja}
\label{tab:spi_pins}
\end{table}

\newpage
\section{Programová implementácia}
Na prácu s displejom bola využité knižnice \texttt{Adafruit-GFX} \cite{Adafruit_GFX} a 
\texttt{Adafruit\_SSD1306} \cite{Adafruit_SSD1306}. Vlastná implementácia programu bola rozdelená do nasledovných modulov:
\begin{itemize}
    \item \textbf{Keypad} – zabezpečuje obsluhu klávesnice a spracovanie vstupov používateľa
    \item \textbf{Display} – realizuje vykresľovanie informácií na displej ako aj pohyb v v správe
    \item \textbf{Buffer} – spravuje pole znakov pre aktuálnu správu a poskytuje základné operácie
\end{itemize}

\subsection{Algoritmus vyčítania klávesnice}
Čítanie maticovej klávesnice je realizované metódou aktívneho skenovania stĺpcov. Každý stĺpec je postupne nastavovaný do aktívneho logického stavu (logická nula). Následne sa vyhodnocujú úrovne na riadkových vstupoch mikrokontroléra. V prípade, že je na niektorom z riadkov detegovaná aktívna logická úroveň (nízka úroveň pri použitých {PULLUP} rezistoroch), je možné jednoznačne určiť pozíciu stlačenej klávesy na základe kombinácie aktuálne skenovaného stĺpca a príslušného riadku. Po detekcii stlačenia klávesy je stĺpec uvedený späť do neaktívneho stavu a algoritmus vracia identifikátor zodpovedajúcej klávesy. Ak počas celého cyklu skenovania nie je detegované žiadne stlačenie, výsledkom je informácia o absencii vstupu. Tento spôsob snímania umožňuje efektívne čítanie klávesnice s minimálnym počtom vstupno-výstupných pinov a je bežne používaný pri maticových klávesniciach.

\begin{algorithm}[h!]
\caption{Čítanie maticovej klávesnice}
\label{alg:keypad_alg2e}

\For{každý stĺpec $c$ klávesnice}{
  nastaviť stĺpec $c$ na logickú 0\;
  \For{každý riadok $r$ klávesnice}{
    \If{riadok $r$ je v logickej 0}{
      detegovaná klávesa na pozícii $[r,c]$\;
      nastaviť stĺpec $c$ na logickú 1\;
      \Return identifikátor klávesy $[r,c]$\;
    }
  }
  nastaviť stĺpec $c$ na logickú 1\;
}
\Return žiadna klávesa\;
\end{algorithm}

\subsection{Logika spracovávania vstupu}
Logika písania znakov je založená na spracovávaní vstupov z maticovej klávesnice a ich mapovaní na konkrétne znaky zobrazované na displeji. Po detekcii stlačenia klávesy je jej identifikátor vyhodnotený riadiacim algoritmom, ktorý rozhoduje o výslednom znaku na základe kontextu aktuálneho vstupu. Implementovaná je podpora viacnásobného stlačenia tej istej klávesy v definovanom časovom intervale, čo umožňuje cyklické prepínanie medzi viacerými znakmi priradenými k jednej klávese. Ak čas medzi jednotlivými stlačeniami prekročí stanovenú hranicu, aktuálny znak je potvrdený a kurzor sa posunie na nasledujúcu pozíciu.\newline

Súčasťou logiky písania je aj rozlišovanie medzi krátkym a dlhým stlačením klávesy. Krátke stlačenie slúži na zadávanie znakov, zatiaľ čo dlhé stlačenie je využité na vyvolanie špeciálnych funkcií, ako je napríklad pohyb po displeji alebo zobrazenie pomocných informácií. Tento prístup umožňuje efektívnu prácu s terminálom aj pri obmedzenom počte vstupných prvkov.

\newpage
\section{Užívateľská príručka}
V tejto kapitole je stručne popísane užívateľské rozhranie, ovládanie terminálu pomocou kláves a písanie správ na termináli, ako aj návod na zobrazenie nápovedy.

\subsection{Popis užívateľského rozhrania}
Displej je rozdelený na dve hlavné časti a to \textbf{Stavový riadok} a \textbf{Textovú oblasť}. V tabuľke \ref{tab:header_info} je popísané rozloženie a vysvetlenie jednotlivých prvkov stavového riadku.\newline

\begin{table}[h!]
    \centering
    \begin{tabular}{l l p{10cm}}
        \hline
        \textbf{Pozícia} & \textbf{Formát} & \textbf{Význam} \\ \hline
        \rule{0pt}{3ex} 
        \textbf{Vľavo} & \texttt{abc}, \texttt{ABC}, \texttt{Abc} & \textbf{Vstupný režim:} Aktuálne zvolený režim veľkosti písmen.\\
        \rule{0pt}{3ex}
        \textbf{Stred} & \texttt{Line X} & \textbf{Riadok:} Číslo riadku, na ktorom sa nachádza kurzor.\\ 
        \rule{0pt}{3ex}
        \textbf{Vpravo} & \texttt{Znaky/Strana} & \textbf{Štatistika:} Prvé číslo udáva počet zostávajúcich znakov z maximálnej dĺžky správy 160 znakov \cite{WikipediaSMS}. Druhé číslo označuje aktuálnu stránku zobrazenú na displeji. \\ \hline
    \end{tabular}
    \caption{Rozloženie a význam prvok stavového riadku}
    \label{tab:header_info}
\end{table}

V textovej oblasti je zobrazovaná aktuálna správa s aktuálnou pozíciou pomocou kurzoru. V rámci správy je možné sa pohybovať a následne editovať alebo mazať jednotlivé znaky.

\subsection{Ovládanie terminálu}
Ovládanie terminálu je realizované pomocou maticovej klávesnice, pričom funkcia jednotlivých kláves sa mení v závislosti od dĺžky stlačenia a aktuálneho kontextu. Hlavné navigačné prvky sú namapované na číselné klávesy:
\begin{itemize}
    \item \textbf{Klávesa 0:} Dlhé podržanie klávesy vymaže celú správu a nastaví kurzor na začiatok správy.
    \item \textbf{Klávesy 2, 4, 6, 8:} Slúžia na pohyb kurzora v štyroch smeroch. Pri dosiahnutí okraja obrazovky dochádza k automatickému zalamovaniu alebo prepnutiu stránky. 
    \item \textbf{Klávesa 5:} Dlhé podržanie klávesy „odošle“ správu a užívateľ môže následne pokračovať v písaní novej správy.
    \item \textbf{Klávesa \#:} Vykonáva funkciu \textit{Backspace} -- zmaže znak pred alebo pod kurzorom.
    \item \textbf{Klávesa *:} Krátkym stlačením cyklicky mení režimy zadávania textu (\texttt{abc} $\rightarrow$ \texttt{ABC} $\rightarrow$ \texttt{Abc}).
\end{itemize}

\subsection{Písanie správ}
Zadávanie textu prebieha metódou \textbf{Multi-tap} \cite{WikipediaMultiTap}, ktorá je typická pre telefóny s numerickou klávesnicou. Opakovaným stláčaním tej istej klávesy v krátkom časovom intervale užívateľ prepína medzi dostupnými znakmi danej klávesy. Znaky pre jednotlivé klávesy sú zobrazené v tabuľke \ref{tab:keysymbols}.\newpage

\begin{table}[h!]
\centering
\begin{tabular}{|c|l|}
\hline
\textbf{Klávesa} & \textbf{Priradené znaky} \\
\hline
0 & \texttt{␣0} \\
1 & \texttt{.,?!1} \\
2 & \texttt{abc2} \\
3 & \texttt{def3} \\
4 & \texttt{ghi4} \\
5 & \texttt{jkl5} \\
6 & \texttt{mno6} \\
7 & \texttt{pqrs7} \\
8 & \texttt{tuv8} \\
9 & \texttt{wxyz9} \\
\hline
\end{tabular} 
\caption{Mapovanie kláves na znaky}
\label{tab:keysymbols}
\end{table}

Špeciálnou funkciou je inteligentný režim \textbf{Smart Case} (indikovaný ako \texttt{Abc}). V tomto režime terminál analyzuje predchádzajúci text, ak deteguje koniec vety, automaticky nastaví nasledujúci znak ako veľké písmeno. Táto funkcionalita výrazne zrýchľuje písanie súvislých správ bez nutnosti manuálneho prepínania režimov.

\subsection{Nápoveda}
Pre užívateľa je k dispozícii v prípade potreby nápoveda s popisom ovládacích prvkov terminálu.
Pre zobrazenie nápovedy je potrebné podržať na klávesnici tlačidlo \texttt{*}, pre zavretie nápovedy a návrat do aktuálnej pozície v správe stačí dané tlačidlo pustiť.\newline

\begin{table}[h!]
    \centering
    \begin{tabular}{r@{:~}l | r@{:~}l}
        \rule{0pt}{2.5ex} 
        0 & Clear & 2 & UP \\
        5 & Send  & 4 & LEFT \\
        \textbf{*} & Mode & 6 & RIGHT \\
        \textbf{\#} & Del & 8 & DOWN \\
    \end{tabular}
    \label{tab:help_menu}
    \caption{Obsah zobrazenej nápovedy}
\end{table}

\newpage
\section{Záver}
V rámci projektu sa podarilo úspešne navrhnúť a implementovať textový terminál s využitím mikrokontroléra \textsc{ESP32}. Hlavným prínosom riešenia je manuálne vyčítanie maticovej klávesnice, ktorá bola realizovaná výhradne prostredníctvom priamej manipulácie s \textsc{GPIO} vstupmi a výstupmi mikrokontroléra, čím bola v plnom rozsahu splnená kľúčová požiadavka zadania na nepoužívanie externých knižníc na prácu s klávesnicou. Na implementáciu aplikácie bolo využité prostredie \texttt{Arduino}, pričom bol kladený dôraz na výslednú kvalitu riešenia. 

\newpage
\printbibliography

\end{document}